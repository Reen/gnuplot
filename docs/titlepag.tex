%
% $Id: titlepag.tex,v 1.31 2011/09/10 06:12:49 sfeam Exp $
%

\ifx\LaTeXe\undefined

% old LaTeX version
% add `,a4' to `toc_entry' to load settings for A4-paper
% see below if you add 11pt or 12pt
   \documentstyle[toc_entr]{article}

\else

% LaTeX2e version
% add `[a4paper]' before `{article}' to load settings for A4-paper
% see below if you add 11pt or 12pt
   \documentclass[twoside]{article}
   \usepackage{toc_entr}
   \usepackage[utf8]{inputenc}

  \usepackage[
% We've given up trying to understand pdftex vs. normal latex.
% It'll just have to sort itself out.  If it doesn't, add your
% own local hyperref options HERE.
        hyperindex,
        bookmarks,
        bookmarksnumbered=true,
        pdftitle={gnuplot documentation},
        pdfauthor={gnuplot},
        pdfsubject={see www.gnuplot.info}
  %     ,pdfcreator={}
  %     ,pdfkeywords={...}
  ]{hyperref}

  \usepackage{fancyhdr}

  \usepackage{makeidx}
  \makeindex

  % This is only needed if you want to embed figures
  


\fi

% The following statements should adjust the default values for
% different papersizes, mainly required for verbatim output
% 30pt are a bit more than really needed
%\addtolength{\textwidth}{30pt}
%\addtolength{\oddsidemargin}{-15pt}
%\addtolength{\evensidemargin}{-15pt}
% Approximately keep the same ratio of width/height
%\addtolength{\textheight}{48pt}
%\addtolength{\topmargin}{-24pt}

% \setlength{\oddsidemargin}{0.5cm}
\setlength{\oddsidemargin}{0.0cm}
\setlength{\evensidemargin}{0.0cm}
\setlength{\topmargin}{-0.5in}
\setlength{\textwidth}{6.50in}
\setlength{\textheight}{9.5in}

\setlength{\parskip}{1ex}
\setlength{\parindent}{0pt}

% For 11pt/12pt options change `\normalsize' to `\small' in
% preverbatim
% every verbatim environment is surrounded by the commands
\newcommand{\preverbatim}{\normalsize\vspace{-2.2ex}}
\newcommand{\postverbatim}{\normalsize\vspace{-0.5ex}}

\adjustarticle

\setcounter{secnumdepth}{5}
\setcounter{tocdepth}{5}


% Read the gnuplot version into macro \gnuplotVersion 
\newread\fileGpVersion
\openin\fileGpVersion VERSION
\ifeof\fileGpVersion\openin\fileGpVersion VERSION.
\ifeof\fileGpVersion\error FATAL: Cannot read file "VERSION"
\fi\fi
\read\fileGpVersion to \gpVersion
\closein\fileGpVersion
\newbox\GpVersion \setbox\GpVersion=\hbox{\gpVersion}
\def\gnuplotVersion{\usebox\GpVersion}


% Layout setup of fancy headings:
\pagestyle{fancy} \headsep=5.5mm \addtolength{\headheight}{7mm}
%\setlength{\headrulewidth}{0.4pt}
\chead{\hyperlink{TableOfContents}{gnuplot \usebox\GpVersion}}
\cfoot{}
\rhead[\leftmark]{\thepage}
\lhead[\thepage]{\leftmark}

\begin{document}

\sloppy
\thispagestyle{empty}
\rule{0in}{1.0in}

  \begin{center}

  {\huge\bf {gnuplot \gpVersion}}\\
  \vspace{3ex}
  {\Large An Interactive Plotting Program}\\

  \vspace{2ex}

  \large
  Thomas Williams \& Colin Kelley\\

  \vspace{2ex}

  Version
    \gnuplotVersion
  organized by: Hans-Bernhard Bröker, Ethan A Merritt, and others\\

   \vspace{2ex}

  Major contributors (alphabetic order):\\

  Hans-Bernhard Bröker,
  John Campbell,\\
  Robert Cunningham,
  David Denholm,\\
  Gershon Elber,
  Roger Fearick,\\
  Carsten Grammes,
  Lucas Hart, \\
  Lars Hecking,
  Péter Juhász, \\
  Thomas Koenig,
  David Kotz,\\
  Ed Kubaitis,
  Russell Lang,\\
  Timothée Lecomte,
  Alexander Lehmann,\\
  Alexander Mai,
  Bastian Märkisch, \\
  Ethan A Merritt,
  Petr Mikulík,\\
  Carsten Steger,
  Shigeharu Takeno,\\
  Tom Tkacik,
  Jos Van der Woude,\\
  James R. Van Zandt,
  Alex Woo,
  Johannes Zellner\\
  Copyright {\copyright} 1986 - 1993, 1998, 2004   Thomas Williams, Colin Kelley\\
  Copyright {\copyright} 2004 - 2011  various authors\\

  \vspace{2ex}

  Mailing list for comments: \verb+gnuplot-info@lists.sourceforge.net+\\
  Mailing list for bug reports: \verb+gnuplot-bugs@lists.sourceforge.net+\\
  Web access (preferred): \verb+http://sourceforge.net/projects/gnuplot+

  \vfill
  This manual was originally prepared by Dick Crawford. \\

  \vspace{2ex}

% 10 September 2011  Version 4.5
 2011  Version 4.5 (cvs)

   \end{center}
\newpage


\hypertarget{TableOfContents}{}
\tableofcontents

\newpage
